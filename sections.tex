%% Sections

%%
%% Use \section{} to divide your presentation into chapters.
%% Each \section{} generates a section frame automatically.
%% The variant \section*{} will not appear in the table of contents and will not generate a section frame
%%
\section*{Motivation}


%%
%% Every slide is written in a frame environment (\begin{frame}...\end{frame}).
%% You should provide a frame title after \begin{frame}.
%%

\begin{frame}{Motivation}
  This \LaTeX{} Beamer template imitates the layout of the old official RWTH presentation template.
\end{frame}

%%
%% Table of contents (automatically collects all \section{} and \subsection{} entries).
%% (run pdflatex multiple times to get all cross references correct)
%%
\begin{frame}[t]{Agenda}
%\begin{multicols}{2}
  \tableofcontents[hideallsubsections]
%\end{multicols}
\end{frame}

\section{Example slides}

\begin{frame}{Titel der Folie, der auch in die zweite Zeile laufen darf.}
  \begin{itemize}
    \item Liste mit Aufzählungszeichen
    \item Liste mit Aufzählungszeichen
      \begin{itemize}
        \item Zweite Ebene
          \begin{itemize}
            \item Optionale dritte Ebene
          \end{itemize}
      \end{itemize}
    \item Liste mit Aufzählungszeichen
    \item Liste mit Aufzählungszeichen
      \begin{itemize}
        \item Zweite Ebene
      \end{itemize}
  \end{itemize}
\end{frame}

\begin{frame}{Noch eine Folie mit Aufzählungen}
  \begin{enumerate}
    \item Erster Punkt
      \begin{enumerate}
        \item Erster Unterpunkt
        \item Zweiter Unterpunkt
      \end{enumerate}
    \item Zweiter Punkt
      \begin{itemize}
        \item Erster regulärer Unterpunkt
        \item Zweiter regulärer Unterpunkt
      \end{itemize}
  \end{enumerate}
\end{frame}

\begin{frame}{Titel der Folie mit Absätzen}
  \begin{itemize}
    \item Lorem ipsum dolor sit amet, consectetur adipisicing elit, sed do eiusmod tempor incididunt ut labore et dolore magna aliqua.
    \item Lorem ipsum dolor sit amet, consectetur adipisicing elit, sed do eiusmod tempor incididunt ut labore et dolore magna aliqua.
    \item Lorem ipsum dolor sit amet, consectetur adipisicing elit, sed do eiusmod tempor incididunt ut labore et dolore magna aliqua.
  \end{itemize}
\end{frame}

\begin{frame}{Folie mit Boxen}
  \begin{block}{Block 1}
    Inhalt der ersten Box
  \end{block}
  \begin{block}{weitere Box}
    Lorem ipsum dolor sit amet, consectetur adipisicing elit, sed do eiusmod tempor incididunt ut labore et dolore magna aliqua.
  \end{block}
\end{frame}



\section{more Examples}

%%
%% If a \subsection{} is available, that one is used as first line of the frame title.
%%
\subsection{a Subsection}

\begin{frame}{How to set columns}
  %%
  %% The columns environment is provided by beamer,
  %% see "texdoc beamer", Sec. 12.7 'Splitting a Frame into Multiple Columns'
  %%
  %% parameter: c - center columns vertically
  %%            t - align columns on the baseline of the first line
  %%                don't use, if a column contains (only) a graphic!
  %%            T - align columns on the top of the first line (ok with graphics)
  %%            b - align columns on the bottom line
  \begin{columns}[T]
    %% Each column must be given a with.
    %% Should be given relative to \textwidth:
    \begin{column}{.45\textwidth}
      \begin{itemize}
        \item Item 1
        \item Item 2
        \item Item 3
        \item Lorem ipsum dolor sit amet, consectetur adipisicing elit, sed do eiusmod tempor incididunt ut labore et dolore magna aliqua.
      \end{itemize}
    \end{column}
    \begin{column}{.45\textwidth}
      %% here, \textwidth is the with of the current column
      \includegraphics[width=.9\textwidth]{pictures/Tux}
    \end{column}
  \end{columns}
\end{frame}


\begin{frame}{How to set columns (2)}
  Lorem ipsum dolor sit amet, consectetur adipisicing elit, sed do eiusmod tempor incididunt ut labore et dolore magna aliqua.
  \begin{columns}[t]
    \begin{column}{.4\textwidth}
      \begin{itemize}
        \item cat
        \item dog
        \item mouse
        \item elephant
      \end{itemize}
    \end{column}
    \begin{column}{.6\textwidth}
      \begin{itemize}
        \item Lorem ipsum dolor sit amet, consectetur adipisicing elit, sed do eiusmod tempor incididunt ut labore et dolore magna aliqua.
        \item Lorem ipsum dolor sit amet, consectetur adipisicing elit, sed do eiusmod tempor incididunt ut labore et dolore magna aliqua.
      \end{itemize}
    \end{column}
  \end{columns}

  Lorem ipsum dolor sit amet, consectetur adipisicing elit, sed do eiusmod tempor incididunt ut labore et dolore magna aliqua.
\end{frame}

\subsection{another subsection}

\begin{frame}
  This frame does not contain a (dedicated) frame title.
\end{frame}

\section{Examples with uncovering}

\begin{frame}{Piecewise uncovering using pause}
  Paragraphs and items can be uncovered easily with \texttt{\textbackslash pause}.
  \pause
  \begin{itemize}
    \item Lorem ipsum dolor sit amet, consectetur adipisicing elit, sed do eiusmod tempor incididunt ut labore et dolore magna aliqua.
      \pause
    \item Lorem ipsum dolor sit amet, consectetur adipisicing elit, sed do eiusmod tempor incididunt ut labore et dolore magna aliqua.
      \pause
    \item Lorem ipsum dolor sit amet, consectetur adipisicing elit, sed do eiusmod tempor incididunt ut labore et dolore magna aliqua.
      \pause
    \item Lorem ipsum dolor sit amet, consectetur adipisicing elit, sed do eiusmod tempor incididunt ut labore et dolore magna aliqua.
  \end{itemize}
\end{frame}

\begin{frame}{Fine grained control}
  \begin{itemize}
    \item<1-> First line
    \item<2>  Second line (appears after first line, disappears again)
    \item<3-4> Third line (appears before firth one)
    \item<4-> Fouth line
    \item<5-> Fifth line
  \end{itemize}
  \onslide<6->{
    Standard behavior with \texttt{\textbackslash onslide} is to keep its place, even if not displayed.
  }

  \only<7>{
    With \texttt{\textbackslash only}, the element does not occupy place. This may lead to shifting other elements.
  }
\end{frame}

\section{Quellcode}

%%
%% This requires the package listings
%%
%% Slides containing lstlisting environments, \lstinline|..| or \verb|..|
%% the option "fragile" must be provided!
%%
\begin{frame}[fragile]{Example-Code: Hallo Welt!}
  \begin{lstlisting}[language=C,gobble=4]
    #include <stdio.h>

    void main (void) {
      printf("Hallo Welt!\n");
    }
  \end{lstlisting}
\end{frame}




\section{TikZ example}

%%
%% This requires the package tikz
%%
%% The animation uses the same overlay specification<2-3> as the items or \onslide
%%
\begin{frame}{animated graphics}

  % define some styles
  \tikzstyle{format} = [draw, thin, fill=blue!20]
  \tikzstyle{medium} = [ellipse, draw, thin, fill=green!20, minimum height=2.5em]

  \begin{figure}
    \begin{tikzpicture}[node distance=3cm, auto,>=latex', thick]
      % We need to set a bounding box first. Otherwise the diagram
      % will change position for each frame.
      \path[use as bounding box] (-1,0) rectangle (10,-2);
      \path[->]<1-> node[format] (tex) {.tex file};
      \path[->]<2-> node[format, right of=tex] (dvi) {.dvi file}
        (tex) edge node {\TeX} (dvi);
      \path[->]<3-> node[format, right of=dvi] (ps) {.ps file}
        node[medium, below of=dvi] (screen) {screen}
        (dvi) edge node {dvips} (ps)
        edge node[swap] {xdvi} (screen);
      \path[->]<4-> node[format, right of=ps] (pdf) {.pdf file}
        node[medium, below of=ps] (print) {printer}
        (ps) edge node {ps2pdf} (pdf)
        edge node[swap] {gs} (screen)
        edge (print);
      \path[->]<5-> (pdf) edge (screen)
        edge (print);
      \path[->, draw]<6-> (tex) -- +(0,1) -| node[near start] {pdf\TeX} (pdf);
    \end{tikzpicture}
  \end{figure}

\end{frame}


%%%%%%%%%%%%%%%%%%%%%%%%%%%%%%%%%%%%%%%%%%%%%%%%%%%%%%%%%%%%%%%%%%%%%%%%%%%%%%%%%%%%%%%%%%%%%%%%%%%%%%%%%%%%%%%%%%%%%%%%%%%%%%%%%%%%%%%%%%%%%%%%
%%%%%%%%%%%%%%%%%%%%%%%%%%%%%%%%%%%%%%%%%%%%%%%%%%%%%%%%%%%%%%%%%%%%%%%%%%%%%%%%%%%%%%%%%%%%%%%%%%%%%%%%%%%%%%%%%%%%%%%%%%%%%%%%%%%%%%%%%%%%%%%%


%\begin{frame}{Titel der Folie, der auch in die zweite Zeile laufen darf.}
%  \begin{itemize}
%    \item Liste mit Aufzählungszeichen
%    \item Liste mit Aufzählungszeichen
%      \begin{itemize}
%        \item Zweite Ebene
%          \begin{itemize}
%            \item Optionale dritte Ebene
%          \end{itemize}
%      \end{itemize}
%    \item Liste mit Aufzählungszeichen
%    \item Liste mit Aufzählungszeichen
%      \begin{itemize}
%        \item Zweite Ebene
%      \end{itemize}
%  \end{itemize}
%\end{frame}
%
%\begin{frame}{Titel der Folie mit Absätzen}
%  \begin{itemize}
%    \item Lorem ipsum dolor sit amet, consectetur adipisicing elit, sed do eiusmod tempor incididunt ut labore et dolore magna aliqua.
%    \item Lorem ipsum dolor sit amet, consectetur adipisicing elit, sed do eiusmod tempor incididunt ut labore et dolore magna aliqua.
%    \item Lorem ipsum dolor sit amet, consectetur adipisicing elit, sed do eiusmod tempor incididunt ut labore et dolore magna aliqua.
%  \end{itemize}
%\end{frame}
%
%\section[TOC-Titel]{Section-Titel}
%
%\begin{frame}{Test}
%  Test
%\end{frame}
%
%\begin{frame}
%  \begin{itemize}
%    \item Punkt 1
%    \item Punkt 2
%    \item Punkt 3
%  \end{itemize}
%\end{frame}
%
%\begin{frame}
%  \begin{tikzpicture}
%    \node[draw] {Hello, world};
%  \end{tikzpicture}
%\end{frame}
%
%\begin{frame}{Optionaler Titel der Folie mit Bild}
%\centerline{%
%  \includegraphics[width=\textwidth]{logos/rwth}
%  }
%\end{frame}

