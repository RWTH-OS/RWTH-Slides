\documentclass[]{beamer}

%%
%% RWTH Beamer Theme (layout based on RWTH PowerPoint template)
%% by Georg Wassen, Lehrstuhl für Betriebssysteme, 2013
%%    wassen@lfbs.rwth-aachen.de
%% with modifications by Gerrit Toehgiono, Mentoring Informatik, 2016-2018
%% to match the "current old" RWTH template
%%
%% The templates are derived from the beamer documentation and the provided templates,
%% hence, the same licence applies:
%%
%% ----------------------------------------------------------------
%% |  This file may be distributed and/or modified                |
%% |                                                              |
%% |  1. under the LaTeX Project Public License and/or            |
%% |  2. under the GNU Public License.                            |
%% |                                                              |
%% |  See the file doc/licenses/LICENSE for more details.         |
%% ----------------------------------------------------------------
%%
%% Version 0.1    20.12.2012    Initial presentation using this theme
%% Version 0.2    08.01.2013    Extracted layout and some example slides
%% Version 0.3    12.01.2013    Improved handling of \subsection (used as frame title),
%%                              Support institute's logo
%% Version 0.4    22.01.2013    Improved spacing between lines in header & footer
%% Version 0.5    01.02.2013    No right sidebar: set width correctly
%% Version 1.0    06.03.2014    Initial Tag on GitHub
%%
%% Version 1.1    31.03.2018    Modifications for CS Mentoring
%%
%% TODO
%%  - copyright of slides?
%%  - count slides with same subsection (needs probably overloading \subsection) and print: 1/N
%%    (needs to count the number of frames in each subsection)

%%%%%%%%%%%%%%%%%%%%%%%%%%%%%%%%%%%%%
%% Select input file encoding:
%%   utf8   - UTF-8, nowadays standard on most operating sytems
%%   latin1 - ISO-8859-1
\usepackage[utf8]{inputenc}

%%%%%%%%%%%%%%%%%%%%%%%%%%%%%%%%%%%%%
%% Select language
%%
\usepackage[ngerman]{babel}        % Deutsch, neue Rechtschreibung
%\usepackage[english]{babel}       % English

\usetheme{rwth}
\usepackage[T1]{fontenc}           % Font encoding (don't change!)
\usepackage{lmodern}               % Select Linux Modern Fonts (don't change)
\usepackage{sansmathfonts}         % Sans fonts in math environments
\usepackage{textcomp}              % fix 'missing font symbols' warning
\renewcommand{\rmdefault}{phv}     % Arial like (Helvetica)
\renewcommand{\sfdefault}{phv}     % Arial like (Helvetica)

%% graphics related packages
\usepackage{graphicx}              % needed to include graphics (don't change)
\usepackage{epstopdf}              % required to include eps files
%\usepackage{svg}                   % include svg files (requires Inkscape)
\usepackage[encoding,filenameencoding=utf8]{grffile} % allow utf8 file names in graphics

%%%%%%%%%%%%%%%%%%%%%%%%%%%%%%%%%%%%%
%% import packages for content
%%
\usepackage{listings}                           % for lstlisting and \lstinline|..|
%% TikZ can be used to /program/ graphics.
\usepackage{tikz}                                % comment-out, if you don't need this.
%% some TikZ-libraries and settings for the examples...
\usetikzlibrary{shadings}           % GW: color gradients
\usetikzlibrary{arrows,calc,positioning,fit,matrix,shadows,chains,arrows,shapes,spy,fadings}
\usepackage{pgfplots}
\usetikzlibrary{pgfplots.units,shapes.symbols,shapes.arrows}
%\usetikzlibrary{pgfplots.external}
%\tikzexternalize[prefix=tmp/]

%% Custom packages and definitions

% Mathematikumgebung
\usepackage{amsmath}
\usepackage{amssymb}
\usepackage{sansmath}

% tabularx -> bessere "tabular"-Umgebung
\usepackage{tabularx}

% zusätzliche Formatbezeichner für die tabularx-Umgebung
\newcolumntype{L}{>{\raggedright\let\newline\\\arraybackslash\hspace{0pt}}X}
\newcolumntype{R}{>{\raggedleft\let\newline\\\arraybackslash\hspace{0pt}}X}
\newcolumntype{C}{>{\centering\let\newline\\\arraybackslash\hspace{0pt}}X}

% center text vertically in tabularx(column)
%\renewcommand{\tabularxcolumn}[1]{>{\large}m{#1}}

% Bessere Tabellenlinien
\usepackage{booktabs}

% Tabellenzeilen für booktabs anpassen -> call on frame with table
\newcommand{\fixbooktabsrowhight}{%
  \setlength{\aboverulesep}{0pt}
  \setlength{\belowrulesep}{0pt}
  \setlength{\extrarowheight}{.5ex}
}

% Zellen über mehrere Zeilen
\usepackage{multirow}

% Source, e.g. for images
\setbeamercolor{framesource}{fg=gray}
\setbeamerfont{framesource}{size=\tiny}

\usepackage[absolute,overlay]{textpos}
\newcommand{\source}[1]{\begin{textblock*}{\linewidth}(1ex,\paperheight-2.75em)
  \begin{beamercolorbox}[left]{framesource}
    \usebeamerfont{framesource}\usebeamercolor[fg]{framesource} Source: {#1}
  \end{beamercolorbox}
\end{textblock*}}

\usepackage{etoolbox}
%% short titles for toc \(sub)section[SHORTTITLE for toc]{LONGTITLE for slide}
\makeatletter
% Insert [short title] for \section in ToC
\patchcmd{\beamer@section}{{#2}{\the\c@page}}{{#1}{\the\c@page}}{}{}
% Insert [short title] for \section in Navigation
\patchcmd{\beamer@section}{{\the\c@section}{\secname}}{{\the\c@section}{#1}}{}{}
% Insert [short title] for \subsection in ToC
\patchcmd{\beamer@subsection}{{#2}{\the\c@page}}{{#1}{\the\c@page}}{}{}
% Insert [short title] for \subsection  in Navigation
\patchcmd{\beamer@subsection}{{\the\c@subsection}{#2}}{{\the\c@subsection}{#1}}{}{}
\makeatother



% include config
%%%%%%%%%%%%%%%%%%%%%%%%%%%%%%%%%%%%%
%% configure title page and author information
%%-------------------------------
%% You can always provide a short version: \title[short]{long title}
%%   title        -- Title of the presentation
%%                   The title appears on the first page and may contain a line break: \\ 
%%                   The short title appears in the footer line
%%   subtitle     -- Appears below the title
%%   titlegraphic -- Currently not supported
%%   author       -- Name of the author(s)
%%   email        -- E-Mail address of author (optional)
%%   institute    -- Name of the institution (e.g. chair)
%%   webaddress   -- Web address (default is www.rwth-aachen.de), displayed on last slide
%%   date         -- Date of the presentation (or use \date to insert the date of the PDF generation)
%%   subject      -- This is only for the PDF meta data
%%   keywords     -- This is only for the PDF meta data
%%   logo         -- header Logo, don't change (given by coporate identity templates)
%%   instlogo     -- Logo of institute/chair, if given: will be shown in foot line
\title[RWTH presentation template]{(In--)Official RWTH presentation template\\with examples}
\subtitle{Subtitle}
%\titlegraphic{}
\author{Max Mustermann}
\email{max.mustermann@rwth-aachen.de} % optionally
\institute{Institut für Mustertechnologie}
%\webaddress{www.informatik.rwth-aachen.de/mentoring} % overrides www.rwth-aachen.de
\date{\today}
\subject{RWTH presentation template}
\keywords{RWTH, Latex Beamer, template}

%\logo{\includegraphics[height=8mm]{logos/logo}} % will replace the default logo

% official institute logo offset correction
%\logo{\vskip-3.5mm\includegraphics[height=12.5mm]{logos/rwth_mentoring_rgb.eps}\hspace{-2mm}} % optionally
\logo{\vskip-2mm\includegraphics[width=45mm]{logos/logo}\hspace{-2mm}} % optionally

% alternative logo position (not recommended)
%\instlogo{\includegraphics[height=10mm]{logos/rwth_mentoring_rgb.eps}} % optionally

%%%%%%%%%%%%%%%%%%%%%%%%%%%%%%%%%%%%%
%% configure template behaviour
%%-------------------------------
%%   secstart -- style of section start
%%               selectable parameters:
%%                 sectitle:  only provides section title
%%                 sectoc:    display section table of contents
%%                 <empty>:   display nothing on section start
\secstart{sectitle}



% disable PDF navigation icons
\setbeamertemplate{navigation symbols}{}

\begin{document}
%%
%% the title slide is generated automatically (at begin document)
%%

% include sections
%% Sections

%%
%% Use \section{} to divide your presentation into chapters.
%% Each \section{} generates a section frame automatically.
%% The variant \section*{} will not appear in the table of contents and will not generate a section frame
%%
\section*{Motivation}


%%
%% Every slide is written in a frame environment (\begin{frame}...\end{frame}).
%% You should provide a frame title after \begin{frame}.
%%

\begin{frame}{Motivation}
  This \LaTeX{} Beamer template imitates the layout of the old official RWTH presentation template.
\end{frame}

%%
%% Table of contents (automatically collects all \section{} and \subsection{} entries).
%% (run pdflatex multiple times to get all cross references correct)
%%
\begin{frame}[t]{Agenda}
%\begin{multicols}{2}
  \tableofcontents[hideallsubsections]
%\end{multicols}
\end{frame}

\section{Example slides}

\begin{frame}{Titel der Folie, der auch in die zweite Zeile laufen darf.}
  \begin{itemize}
    \item Liste mit Aufzählungszeichen
    \item Liste mit Aufzählungszeichen
      \begin{itemize}
        \item Zweite Ebene
          \begin{itemize}
            \item Optionale dritte Ebene
          \end{itemize}
      \end{itemize}
    \item Liste mit Aufzählungszeichen
    \item Liste mit Aufzählungszeichen
      \begin{itemize}
        \item Zweite Ebene
      \end{itemize}
  \end{itemize}
\end{frame}

\begin{frame}{Noch eine Folie mit Aufzählungen}
  \begin{enumerate}
    \item Erster Punkt
      \begin{enumerate}
        \item Erster Unterpunkt
        \item Zweiter Unterpunkt
      \end{enumerate}
    \item Zweiter Punkt
      \begin{itemize}
        \item Erster regulärer Unterpunkt
        \item Zweiter regulärer Unterpunkt
      \end{itemize}
  \end{enumerate}
\end{frame}

\begin{frame}{Titel der Folie mit Absätzen}
  \begin{itemize}
    \item Lorem ipsum dolor sit amet, consectetur adipisicing elit, sed do eiusmod tempor incididunt ut labore et dolore magna aliqua.
    \item Lorem ipsum dolor sit amet, consectetur adipisicing elit, sed do eiusmod tempor incididunt ut labore et dolore magna aliqua.
    \item Lorem ipsum dolor sit amet, consectetur adipisicing elit, sed do eiusmod tempor incididunt ut labore et dolore magna aliqua.
  \end{itemize}
\end{frame}

\begin{frame}{Folie mit Boxen}
  \begin{block}{Block 1}
    Inhalt der ersten Box
  \end{block}
  \begin{block}{weitere Box}
    Lorem ipsum dolor sit amet, consectetur adipisicing elit, sed do eiusmod tempor incididunt ut labore et dolore magna aliqua.
  \end{block}
\end{frame}



\section{more Examples}

%%
%% If a \subsection{} is available, that one is used as first line of the frame title.
%%
\subsection{a Subsection}

\begin{frame}{How to set columns}
  %%
  %% The columns environment is provided by beamer,
  %% see "texdoc beamer", Sec. 12.7 'Splitting a Frame into Multiple Columns'
  %%
  %% parameter: c - center columns vertically
  %%            t - align columns on the baseline of the first line
  %%                don't use, if a column contains (only) a graphic!
  %%            T - align columns on the top of the first line (ok with graphics)
  %%            b - align columns on the bottom line
  \begin{columns}[T]
    %% Each column must be given a with.
    %% Should be given relative to \textwidth:
    \begin{column}{.45\textwidth}
      \begin{itemize}
        \item Item 1
        \item Item 2
        \item Item 3
        \item Lorem ipsum dolor sit amet, consectetur adipisicing elit, sed do eiusmod tempor incididunt ut labore et dolore magna aliqua.
      \end{itemize}
    \end{column}
    \begin{column}{.45\textwidth}
      %% here, \textwidth is the with of the current column
      \includegraphics[width=.9\textwidth]{pictures/Tux}
    \end{column}
  \end{columns}
\end{frame}


\begin{frame}{How to set columns (2)}
  Lorem ipsum dolor sit amet, consectetur adipisicing elit, sed do eiusmod tempor incididunt ut labore et dolore magna aliqua.
  \begin{columns}[t]
    \begin{column}{.4\textwidth}
      \begin{itemize}
        \item cat
        \item dog
        \item mouse
        \item elephant
      \end{itemize}
    \end{column}
    \begin{column}{.6\textwidth}
      \begin{itemize}
        \item Lorem ipsum dolor sit amet, consectetur adipisicing elit, sed do eiusmod tempor incididunt ut labore et dolore magna aliqua.
        \item Lorem ipsum dolor sit amet, consectetur adipisicing elit, sed do eiusmod tempor incididunt ut labore et dolore magna aliqua.
      \end{itemize}
    \end{column}
  \end{columns}

  Lorem ipsum dolor sit amet, consectetur adipisicing elit, sed do eiusmod tempor incididunt ut labore et dolore magna aliqua.
\end{frame}

\subsection{another subsection}

\begin{frame}
  This frame does not contain a (dedicated) frame title.
\end{frame}

\section{Examples with uncovering}

\begin{frame}{Piecewise uncovering using pause}
  Paragraphs and items can be uncovered easily with \texttt{\textbackslash pause}.
  \pause
  \begin{itemize}
    \item Lorem ipsum dolor sit amet, consectetur adipisicing elit, sed do eiusmod tempor incididunt ut labore et dolore magna aliqua.
      \pause
    \item Lorem ipsum dolor sit amet, consectetur adipisicing elit, sed do eiusmod tempor incididunt ut labore et dolore magna aliqua.
      \pause
    \item Lorem ipsum dolor sit amet, consectetur adipisicing elit, sed do eiusmod tempor incididunt ut labore et dolore magna aliqua.
      \pause
    \item Lorem ipsum dolor sit amet, consectetur adipisicing elit, sed do eiusmod tempor incididunt ut labore et dolore magna aliqua.
  \end{itemize}
\end{frame}

\begin{frame}{Fine grained control}
  \begin{itemize}
    \item<1-> First line
    \item<2>  Second line (appears after first line, disappears again)
    \item<3-4> Third line (appears before firth one)
    \item<4-> Fouth line
    \item<5-> Fifth line
  \end{itemize}
  \onslide<6->{
    Standard behavior with \texttt{\textbackslash onslide} is to keep its place, even if not displayed.
  }

  \only<7>{
    With \texttt{\textbackslash only}, the element does not occupy place. This may lead to shifting other elements.
  }
\end{frame}

\section{Quellcode}

%%
%% This requires the package listings
%%
%% Slides containing lstlisting environments, \lstinline|..| or \verb|..|
%% the option "fragile" must be provided!
%%
\begin{frame}[fragile]{Example-Code: Hallo Welt!}
  \begin{lstlisting}[language=C,gobble=4]
    #include <stdio.h>

    void main (void) {
      printf("Hallo Welt!\n");
    }
  \end{lstlisting}
\end{frame}




\section{TikZ example}

%%
%% This requires the package tikz
%%
%% The animation uses the same overlay specification<2-3> as the items or \onslide
%%
\begin{frame}{animated graphics}

  % define some styles
  \tikzstyle{format} = [draw, thin, fill=blue!20]
  \tikzstyle{medium} = [ellipse, draw, thin, fill=green!20, minimum height=2.5em]

  \begin{figure}
    \begin{tikzpicture}[node distance=3cm, auto,>=latex', thick]
      % We need to set a bounding box first. Otherwise the diagram
      % will change position for each frame.
      \path[use as bounding box] (-1,0) rectangle (10,-2);
      \path[->]<1-> node[format] (tex) {.tex file};
      \path[->]<2-> node[format, right of=tex] (dvi) {.dvi file}
        (tex) edge node {\TeX} (dvi);
      \path[->]<3-> node[format, right of=dvi] (ps) {.ps file}
        node[medium, below of=dvi] (screen) {screen}
        (dvi) edge node {dvips} (ps)
        edge node[swap] {xdvi} (screen);
      \path[->]<4-> node[format, right of=ps] (pdf) {.pdf file}
        node[medium, below of=ps] (print) {printer}
        (ps) edge node {ps2pdf} (pdf)
        edge node[swap] {gs} (screen)
        edge (print);
      \path[->]<5-> (pdf) edge (screen)
        edge (print);
      \path[->, draw]<6-> (tex) -- +(0,1) -| node[near start] {pdf\TeX} (pdf);
    \end{tikzpicture}
  \end{figure}

\end{frame}


%%%%%%%%%%%%%%%%%%%%%%%%%%%%%%%%%%%%%%%%%%%%%%%%%%%%%%%%%%%%%%%%%%%%%%%%%%%%%%%%%%%%%%%%%%%%%%%%%%%%%%%%%%%%%%%%%%%%%%%%%%%%%%%%%%%%%%%%%%%%%%%%
%%%%%%%%%%%%%%%%%%%%%%%%%%%%%%%%%%%%%%%%%%%%%%%%%%%%%%%%%%%%%%%%%%%%%%%%%%%%%%%%%%%%%%%%%%%%%%%%%%%%%%%%%%%%%%%%%%%%%%%%%%%%%%%%%%%%%%%%%%%%%%%%


%\begin{frame}{Titel der Folie, der auch in die zweite Zeile laufen darf.}
%  \begin{itemize}
%    \item Liste mit Aufzählungszeichen
%    \item Liste mit Aufzählungszeichen
%      \begin{itemize}
%        \item Zweite Ebene
%          \begin{itemize}
%            \item Optionale dritte Ebene
%          \end{itemize}
%      \end{itemize}
%    \item Liste mit Aufzählungszeichen
%    \item Liste mit Aufzählungszeichen
%      \begin{itemize}
%        \item Zweite Ebene
%      \end{itemize}
%  \end{itemize}
%\end{frame}
%
%\begin{frame}{Titel der Folie mit Absätzen}
%  \begin{itemize}
%    \item Lorem ipsum dolor sit amet, consectetur adipisicing elit, sed do eiusmod tempor incididunt ut labore et dolore magna aliqua.
%    \item Lorem ipsum dolor sit amet, consectetur adipisicing elit, sed do eiusmod tempor incididunt ut labore et dolore magna aliqua.
%    \item Lorem ipsum dolor sit amet, consectetur adipisicing elit, sed do eiusmod tempor incididunt ut labore et dolore magna aliqua.
%  \end{itemize}
%\end{frame}
%
%\section[TOC-Titel]{Section-Titel}
%
%\begin{frame}{Test}
%  Test
%\end{frame}
%
%\begin{frame}
%  \begin{itemize}
%    \item Punkt 1
%    \item Punkt 2
%    \item Punkt 3
%  \end{itemize}
%\end{frame}
%
%\begin{frame}
%  \begin{tikzpicture}
%    \node[draw] {Hello, world};
%  \end{tikzpicture}
%\end{frame}
%
%\begin{frame}{Optionaler Titel der Folie mit Bild}
%\centerline{%
%  \includegraphics[width=\textwidth]{logos/rwth}
%  }
%\end{frame}



\end{document}
