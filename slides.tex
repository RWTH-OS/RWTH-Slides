\documentclass[]{beamer}
\usetheme{rwth}

%%
%% RWTH Beamer Theme (layout based on RWTH PowerPoint template)
%% by Georg Wassen, Lehrstuhl für Betriebssysteme
%%    wassen@lfbs.rwth-aachen.de
%%
%% Version 0.1    20.12.2012    initial presentation using this theme
%% Version 0.2    08.01.2013    extracted layout and some example slides
%%
%% TODO
%%  - logo on title page and last page (contact info) is not in the same place as in the other pages
%%  - include e-mail address auf author for last page (contact info)
%%  - set www address for last page (contact info), allow inserting institute's web address
%%  - set "Vielen Dank" (on last page) according to language setting (babel)
%%  - copyright of slides?


%%%%%%%%%%%%%%%%%%%%%%%%%%%%%%%%%%%%%
%% Select input file encoding:
%%   utf8   - UTF-8, nowadays standard on most operating sytems
%%   latin1 - ISO-8859-1
\usepackage[utf8]{inputenc}                 


%%%%%%%%%%%%%%%%%%%%%%%%%%%%%%%%%%%%%
%% Select language
%%
\usepackage[ngerman]{babel}
%\usepackage[english]{babel}

\usepackage[T1]{fontenc}                        % Font encoding (don't change!)
\usepackage{lmodern}                            % Select Linux Modern Fonts (don't change)
\usepackage{graphicx}                           % needed to include graphics (don't change)


%% TikZ can be used to /program/ graphics.
\usepackage{tikz}                                % comment-out, if you don't need this.
%% some TikZ-libraries and settings for the examples...
%\usetikzlibrary{shapes,arrows}                  
\usetikzlibrary{shadings}           % GW: color gradients
\usetikzlibrary{calc,positioning,fit,matrix,shadows,chains,arrows,shapes,spy,fadings}
\usepackage{pgfplots}
\usetikzlibrary{pgfplots.units,shapes.symbols,shapes.arrows}
%\usetikzlibrary{pgfplots.external}
%\tikzexternalize[prefix=tmp/]



%%%%%%%%%%%%%%%%%%%%%%%%%%%%%%%%%%%%%
%% configure title page and author information
%%-------------------------------
%% You can always provide a short version: \title[short]{long title}
%%   title        -- title of the presentation
%%   subtitle     -- appears below the title
%%   titlegraphic -- currently not supported
%%   author       -- name of the author(s)
%%   institute    -- name of the institution (e.g. chair)
%%   date         -- date of the presentation (or use \date to insert the date of the PDF generation)
%%   subject      -- this is only for the PDF meta data
%%   keywords     -- this is only for the PDF meta data
%%   logo         -- logo, don't change (given by coporate identity templates)
\title[NovaCarts-Projekt]{Projektbericht\\MicroNova-Carts}
\subtitle{wissenschaftlicher Diskurs}
%\titlegraphic{}            
\author{Georg Wassen}
\institute{Lehrstuhl für Betriebssysteme}
%\date[09.03.2012]{9. März 2011}
\date{20.12.2012}
\subject{Projektbericht MicroNova-Carts}          
\keywords{Hartus}           
\logo{\includegraphics[height=8mm]{logos/rwth}}     % TODO : move to style files.



\begin{document}

%%
%% the title slide is generated automatically (at begin document)
%%

%%
%% Use \section{} to divide your presentation into chapters.
%% The variant \section*{} will not appear in the table of contents and will not generate a section frame
%%
\section*{Motivation}


%%
%% Every slide is written in a frame environment (\begin{frame}...\end{frame}).
%% You should provide a frame title after \begin{frame}.
%%

\begin{frame}{Motivation}
  This \LaTeX{} Beamer template imitates the layout of the official RWTH presentation template.
\end{frame}

%%
%% Table of contents (automatically collects all \section{} and \subsection{} entries).
%% (run pdflatex multiple times to get all cross references correct)
%%
\begin{frame}{Agenda}
  \tableofcontents
\end{frame}

\section{Example slides}

\begin{frame}{Titel der Folie, der auch in die zweite Zeile laufen darf.}
  \begin{itemize}
    \item Liste mit Aufzählungszeichen
    \item Liste mit Aufzählungszeichen
      \begin{itemize}
        \item Zweite Ebene
          \begin{itemize}
            \item Optionale dritte Ebene
          \end{itemize}
      \end{itemize}
    \item Liste mit Aufzählungszeichen
    \item Liste mit Aufzählungszeichen
      \begin{itemize}
        \item Zweite Ebene
      \end{itemize}
  \end{itemize}
\end{frame}

\begin{frame}{Titel der Folie mit Absätzen}
  \begin{itemize}
    \item Lorem ipsum dolor sit amet, consectetur adipisicing elit, sed do eiusmod tempor incididunt ut labore et dolore magna aliqua.
    \item Lorem ipsum dolor sit amet, consectetur adipisicing elit, sed do eiusmod tempor incididunt ut labore et dolore magna aliqua.
    \item Lorem ipsum dolor sit amet, consectetur adipisicing elit, sed do eiusmod tempor incididunt ut labore et dolore magna aliqua.
  \end{itemize}
\end{frame}

\begin{frame}{Folie mit Boxen}
  \begin{block}{Block 1}
    Inhalt der ersten Box
  \end{block}
  \begin{block}{weitere Box}
    Lorem ipsum dolor sit amet, consectetur adipisicing elit, sed do eiusmod tempor incididunt ut labore et dolore magna aliqua.
  \end{block}
\end{frame}



\section{more Examples}

%% currently, subsections only appear in the logical table of contents of the PDF viewer, not on any slide!
\subsection{a Subsection}
%% You may use the frame titles to subdivide your presentation

\begin{frame}{How to set columns}
  %%
  %% The columns environment is provided by beamer, 
  %% see "texdoc beamer", Sec. 12.7 'Splitting a Frame into Multiple Columns'
  %%
  %% parameter: c - center columns vertically
  %%            t - align columns on the baseline of the first line
  %%                don't use, if a column contains (only) a graphic! 
  %%            T - align columns on the top of the first line (ok with graphics)
  %%            b - align columns on the bottom line
  \begin{columns}[T]
    %% Each column must be given a with.
    %% Should be given relative to \textwidth:
    \begin{column}{.45\textwidth}
      \begin{itemize}
        \item Item 1
        \item Item 2
        \item Item 3
        \item Lorem ipsum dolor sit amet, consectetur adipisicing elit, sed do eiusmod tempor incididunt ut labore et dolore magna aliqua. 
      \end{itemize}
    \end{column}
    \begin{column}{.45\textwidth}
      %% here, \textwidth is the with of the current column
      \includegraphics[width=.9\textwidth]{pictures/Tux}
    \end{column}
  \end{columns}
\end{frame}

\subsection{another Subsection}

\begin{frame}{How to set columns (2)}
  Lorem ipsum dolor sit amet, consectetur adipisicing elit, sed do eiusmod tempor incididunt ut labore et dolore magna aliqua. 
  \begin{columns}[t]
    \begin{column}{.4\textwidth}
      \begin{itemize}
        \item cat
        \item dog
        \item mouse
        \item elephant
      \end{itemize}
    \end{column}
    \begin{column}{.6\textwidth}
      \begin{itemize}
        \item Lorem ipsum dolor sit amet, consectetur adipisicing elit, sed do eiusmod tempor incididunt ut labore et dolore magna aliqua. 
        \item Lorem ipsum dolor sit amet, consectetur adipisicing elit, sed do eiusmod tempor incididunt ut labore et dolore magna aliqua. 
      \end{itemize}
    \end{column}
  \end{columns}

  Lorem ipsum dolor sit amet, consectetur adipisicing elit, sed do eiusmod tempor incididunt ut labore et dolore magna aliqua. 
\end{frame}

%\section[NovaCarts]{Einführung in die Software NovaCarts} 
%\include{sec1_NovaCarts}

\section{Examples with uncovering}

\begin{frame}{Piecewise uncovering using pause}
  Paragraphs and items can be uncovered easily with \texttt{\textbackslash pause}.
  \pause
  \begin{itemize}
    \item Lorem ipsum dolor sit amet, consectetur adipisicing elit, sed do eiusmod tempor incididunt ut labore et dolore magna aliqua. 
      \pause
    \item Lorem ipsum dolor sit amet, consectetur adipisicing elit, sed do eiusmod tempor incididunt ut labore et dolore magna aliqua. 
      \pause
    \item Lorem ipsum dolor sit amet, consectetur adipisicing elit, sed do eiusmod tempor incididunt ut labore et dolore magna aliqua. 
      \pause
    \item Lorem ipsum dolor sit amet, consectetur adipisicing elit, sed do eiusmod tempor incididunt ut labore et dolore magna aliqua. 
  \end{itemize}
\end{frame}

\begin{frame}{Fine grained control}
  \begin{itemize}
    \item<1-> First line
    \item<2>  Second line (appears after first line, disappears again)
    \item<3-4> Third line (appears before firth one)
    \item<4-> Fouth line
    \item<5-> Fifth line
  \end{itemize}
  \onslide<6->{
    Standard behavior is to keep its place, even if not displayed.
  }

  \only<7>{
    With \texttt{\textbackslash only}, the element does not occupy place. This may lead to shifting other elements.
  }
\end{frame}



%%
%% the final slide (contact information) is generated automatically (at end document)
%%

\end{document}


%%%%%%%%%%%%%%%%%%%%%%%%%%%%%%%%%%%%%%%%%%%%%%%%%%%%%%%%%%%%%%%%%%%%%%%%%%%%%%%%%%%%%%%%%%%%%%%%%%%%%%%%%%%%%%%%%%%%%%%%%%%%%%%%%%%%%%%%%%%%%%%%
%%%%%%%%%%%%%%%%%%%%%%%%%%%%%%%%%%%%%%%%%%%%%%%%%%%%%%%%%%%%%%%%%%%%%%%%%%%%%%%%%%%%%%%%%%%%%%%%%%%%%%%%%%%%%%%%%%%%%%%%%%%%%%%%%%%%%%%%%%%%%%%%


\begin{frame}{Titel der Folie, der auch in die zweite Zeile laufen darf.}
  \begin{itemize}
    \item Liste mit Aufzählungszeichen
    \item Liste mit Aufzählungszeichen
      \begin{itemize}
        \item Zweite Ebene
          \begin{itemize}
            \item Optionale dritte Ebene
          \end{itemize}
      \end{itemize}
    \item Liste mit Aufzählungszeichen
    \item Liste mit Aufzählungszeichen
      \begin{itemize}
        \item Zweite Ebene
      \end{itemize}
  \end{itemize}
\end{frame}

\begin{frame}{Titel der Folie mit Absätzen}
  \begin{itemize}
    \item Lorem ipsum dolor sit amet, consectetur adipisicing elit, sed do eiusmod tempor incididunt ut labore et dolore magna aliqua.
    \item Lorem ipsum dolor sit amet, consectetur adipisicing elit, sed do eiusmod tempor incididunt ut labore et dolore magna aliqua.
    \item Lorem ipsum dolor sit amet, consectetur adipisicing elit, sed do eiusmod tempor incididunt ut labore et dolore magna aliqua.
  \end{itemize}
\end{frame}

\section[Motivation]{Motivation}

\begin{frame}{Test}
  Test
\end{frame}

\begin{frame}
  \begin{itemize}
    \item Punkt 1
    \item Punkt 2
    \item Punkt 3
  \end{itemize}
\end{frame}

\begin{frame}
  \begin{tikzpicture}
    \node[draw] {Hello, world};
  \end{tikzpicture}
\end{frame}

\begin{frame}{Optionaler Titel der Folie mit Bild}
  \includegraphics[width=\textwidth]{logos/rwth}
\end{frame}

\end{document}
